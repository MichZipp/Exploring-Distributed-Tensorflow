\documentclass[journal]{IEEEtran}
\usepackage{blindtext}
\let\labelindent\relax
\usepackage[inline]{enumitem}
\usepackage{graphicx}
\usepackage[acronym,toc,shortcuts]{glossaries}
\usepackage{subcaption}
\usepackage[bookmarksopen, bookmarksdepth=2, breaklinks=true]{hyperref}
\usepackage[official]{eurosym}
\usepackage{listings}

% *** GRAPHICS RELATED PACKAGES ***
%
\ifCLASSINFOpdf
\else
\fi


\newacronym{mbaas}{MBaaS}{Mobile Backend as a Service}

\hyphenation{op-tical net-works semi-conduc-tor}


\begin{document}
\title{Distributed Tensorflow}

\author{\begin{center}
 Michael Zipperle \\ 
 \textit{259564} \\
 Michael.Zipperle@hs-furtwangen.de \\
\end{center}}%
        
% The paper headers
\markboth{Hochschule Furtwangen - Internet of Things, June 2018}{}

% make the title area
\maketitle


\begin{abstract}
%\boldmath


\end{abstract}

% Note that keywords are not normally used for peerreview papers.
%\begin{IEEEkeywords}
%IEEEtran, journal, \LaTeX, paper, template.
%\end{IEEEkeywords}

% For peerreview papers, this IEEEtran command inserts a page break and
% creates the second title. It will be ignored for other modes.
\IEEEpeerreviewmaketitle


% *** START OF SECTIONS ***--------------------------------------------

\section{Einführung}
Eine \ac{ki} wird heutzutage immer mehr eingesetzt, um die Interaktion zwischen Mensch und Maschine zu bessern. Zusätzliche ermöglicht eine \ac{ki} vollkommen neue Möglichkeiten in Bereichen wie Robotik, Daten Analyse, Gesundheitswesen, Autonomes Fahren und vielen mehr. Doch was steckt hinter einer künstlichen Intelligenz? Eine künstliche Intelligenz versucht durch maschinelles Lernen die menschliche Wahrnehmung und das menschliche Handel durch Maschinen nachzubilden. Bei der \ac{ki}-Forschung gibt es viele Verbindung zur Neurologie und Psychologie, den das menschliche Denken muss erforscht und verstanden werden, bevor Maschinen dies nachahmen können. Bis heute ist dies noch nicht annähernd gelungen, die meisten \ac{ki} beschränken sich auf einen bestimmten Teilbereich und sind somit optimiert für einen bestimmten Anwendungsfall \cite{PlanetWissenKI}. \newline

Bei einer \ac{ki} kommt meist maschinelles Lernen zum Einsatz, wobei von erhobenen Daten gelernt wird, um dann Entscheidung treffen zu können. Es gibt eine Reihe an vordefinierten Algorithmen die an sogenannte Entscheidungsbäume erinnern. Diese Algorithmen sind meist nicht dynamisch genug, um mehrere Variablen verarbeiten zu können. Mit der Einführung von Deep Learning als Zweig des maschinellen Lernens ist dieses Problem größtenteils behoben. Durch Deep Learning kann ein kontinuierlicher Lernprozesse umgesetzt werden, der sich stetig an neue Situation anpasst. Dabei basiert Deep Learning auf der Analyse von Big Data und gräbt sich durch riesige Mengen an Daten aus Datenbank, Internetquellen und mehr. Es nutzt neuronale Netze, um der Denkweise eines Menschen möglichst gut nachzuahmen. Deep Learning war bis vor wenigen Jahren nur von wenig Unternehmen genutzt, da der Einsatz viel Ressourcen benötigt und somit relativ teuer war. Durch den Fortschritt der Technik und der besseren Unterstützung der \ac{gpu} hat sich Deep Learning durchgesetzt und kommt in heutigen Durchbrüchen wie autonomes Fahren, maschinelle Übersetzung, Spracherkennung und vieles mehr zum Einsatz \cite{BigDataInsiderDeepLearning}. \newline

Aktuell stellt der Lernprozess in Echtzeit eine große Herausforderung dar, oftmals benötigt der Lernprozess mehrere Stunden bzw. Tage. Für einige Anwendungsfälle reicht dies nicht aus und es wird ein Lernprozess nahe zu in Echtzeit benötigt. Es gibt bereits zahlreiche Deep Learning Frameworks, welche in diesem Arikel aufgezeigt werden. Sehr beliebt ist aktuell das Framework Tensorflow von Google. Dieses Framework wird im Laufe dieses Artikel genauer untersucht und es wird anhand eines Tutorials die Einrichtung einer verteilten Tensorflow Umgebung aufgezeigt. Durch diese verteilte Umgebung kann die Performance des Lernprozess verbessert werden.  

\section{Zusammenfassung}
 Eine \ac{ki} benötigt viel Ressourcen, um eine gute Performance zu gewährleisten. Es gibt heutzutage einige Deep Learning Frameworks, mit der eine \ac{ki} implementiert werden kann. Am Anfang des Artikels werden einige Frameworks vorgestellt. Im Detail wird dann auf das Framework TensorFlow von
 Google eingegangen, dabei werden die unterstützen Plattformversionen und Möglichkeiten, wie TensorFlow auf einem mobilen Endgerät eingesetzt werden kann, beschrieben. Meist reicht eine Instanz nicht aus, um die Anforderungen einer Anwendungen zu erfüllen. Deshalb gibt es die Möglichkeit, mehrere Instanzen zu nutzen, um die Performance des Frameworks zu verbessern. Dazu wird zuerst eine Wissensgrundlage geschaffen, indem die Architektur einer verteilten TensorFlow Umgebung aufgezeigt wird. Anschließend wir erläutert, welche Rolle dabei ein Arbeiter und Parameterserver spielt. Anhand eines Tutorials werden die wichtigsten Schritte für das Aufsetzten einer solchen Umgebung beschrieben, wobei eine kleine Beispielanwendung entwickelt wurde. Diese wurde genutzt, um eine Evaluierung durchzuführen, welche bestätigt, dass der Einsatz von mehreren
 Arbeitern die Performance des Frameworks deutlich verbessert. Dabei wurden verschiedene Konfiguration, unterschiedliche Anzahl an Arbeiter und Parameterserver, für die TensorFlow Umgebung genutzt. Die Performance einer \ac{ki} wird auch in Zukunft eine Herausforderung darstellen, denn die Einsatzgebiete werden immer komplexer und breitgefächerter.  

% *** END OF SECTIONS ***---------------------------------------------

% needed in second column of first page if using \IEEEpubid
%\IEEEpubidadjcol

% An example of a floating figure using the graphicx package.
% Note that \label must occur AFTER (or within) \caption.
% For figures, \caption should occur after the \includegraphics.
% Note that IEEEtran v1.7 and later has special internal code that
% is designed to preserve the operation of \label within \caption
% even when the captionsoff option is in effect. However, because
% of issues like this, it may be the safest practice to put all your
% \label just after \caption rather than within \caption{}.
%
% Reminder: the "draftcls" or "draftclsnofoot", not "draft", class
% option should be used if it is desired that the figures are to be
% displayed while in draft mode.
%
%\begin{figure}[!t]
%\centering
%\includegraphics[width=2.5in]{myfigure}
% where an .eps filename suffix will be assumed under latex, 
% and a .pdf suffix will be assumed for pdflatex; or what has been declared
% via \DeclareGraphicsExtensions.
%\caption{Simulation Results}
%\label{fig_sim}
%\end{figure}

% Note that IEEE typically puts floats only at the top, even when this
% results in a large percentage of a column being occupied by floats.


% An example of a double column floating figure using two subfigures.
% (The subfig.sty package must be loaded for this to work.)
% The subfigure \label commands are set within each subfloat command, the
% \label for the overall figure must come after \caption.
% \hfil must be used as a separator to get equal spacing.
% The subfigure.sty package works much the same way, except \subfigure is
% used instead of \subfloat.
%
%\begin{figure*}[!t]
%\centerline{\subfloat[Case I]\includegraphics[width=2.5in]{subfigcase1}%
%\label{fig_first_case}}
%\hfil
%\subfloat[Case II]{\includegraphics[width=2.5in]{subfigcase2}%
%\label{fig_second_case}}}
%\caption{Simulation results}
%\label{fig_sim}
%\end{figure*}
%
% Note that often IEEE papers with subfigures do not employ subfigure
% captions (using the optional argument to \subfloat), but instead will
% reference/describe all of them (a), (b), etc., within the main caption.


% An example of a floating table. Note that, for IEEE style tables, the 
% \caption command should come BEFORE the table. Table text will default to
% \footnotesize as IEEE normally uses this smaller font for tables.
% The \label must come after \caption as always.
%
%\begin{table}[!t]
%% increase table row spacing, adjust to taste
%\renewcommand{\arraystretch}{1.3}
% if using array.sty, it might be a good idea to tweak the value of
% \extrarowheight as needed to properly center the text within the cells
%\caption{An Example of a Table}
%\label{table_example}
%\centering
%% Some packages, such as MDW tools, offer better commands for making tables
%% than the plain LaTeX2e tabular which is used here.
%\begin{tabular}{|c||c|}
%\hline
%One & Two\\
%\hline
%Three & Four\\
%\hline
%\end{tabular}
%\end{table}


% Note that IEEE does not put floats in the very first column - or typically
% anywhere on the first page for that matter. Also, in-text middle ("here")
% positioning is not used. Most IEEE journals use top floats exclusively.
% Note that, LaTeX2e, unlike IEEE journals, places footnotes above bottom
% floats. This can be corrected via the \fnbelowfloat command of the
% stfloats package.









% if have a single appendix:
%\appendix[Proof of the Zonklar Equations]
% or
%\appendix  % for no appendix heading
% do not use \section anymore after \appendix, only \section*
% is possibly needed

% use appendices with more than one appendix
% then use \section to start each appendix
% you must declare a \section before using any
% \subsection or using \label (\appendices by itself
% starts a section numbered zero.)
%


% use section* for acknowledgement
%\section*{Acknowledgment}


%The authors would like to thank...


% Can use something like this to put references on a page
% by themselves when using endfloat and the captionsoff option.
\ifCLASSOPTIONcaptionsoff
  \newpage
\fi



% trigger a \newpage just before the given reference
% number - used to balance the columns on the last page
% adjust value as needed - may need to be readjusted if
% the document is modified later
%\IEEEtriggeratref{8}
% The "triggered" command can be changed if desired:
%\IEEEtriggercmd{\enlargethispage{-5in}}

% references section

% can use a bibliography generated by BibTeX as a .bbl file
% BibTeX documentation can be easily obtained at:
% http://www.ctan.org/tex-archive/biblio/bibtex/contrib/doc/
% The IEEEtran BibTeX style support page is at:
% http://www.michaelshell.org/tex/ieeetran/bibtex/
%\bibliographystyle{IEEEtran}
% argument is your BibTeX string definitions and bibliography database(s)
%\bibliography{IEEEabrv,../bib/paper}
%
% <OR> manually copy in the resultant .bbl file
% set second argument of \begin to the number of references
% (used to reserve space for the reference number labels box)
\begin{thebibliography}{1}
\bibitem{AmazonLambda}
"Amazon Lambda"
\url{https://aws.amazon.com/de/lambda/} 
Accessed 18.06.2018
\end{thebibliography}

% biography section
% 
% If you have an EPS/PDF photo (graphicx package needed) extra braces are
% needed around the contents of the optional argument to biography to prevent
% the LaTeX parser from getting confused when it sees the complicated
% \includegraphics command within an optional argument. (You could create
% your own custom macro containing the \includegraphics command to make things
% simpler here.)
%\begin{biography}[{\includegraphics[width=1in,height=1.25in,clip,keepaspectratio]{mshell}}]{Michael Shell}

% You can push biographies down or up by placing
% a \vfill before or after them. The appropriate
% use of \vfill depends on what kind of text is
% on the last page and whether or not the columns
% are being equalized.

%\vfill

% Can be used to pull up biographies so that the bottom of the last one
% is flush with the other column.
%\enlargethispage{-5in}



% that's all folks
\end{document}


