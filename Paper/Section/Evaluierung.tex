\section{Evaluierung}
Die Nutzung einer verteilten TensorFlow Umgebung zeigt deutliche Performance Vorteile. Bei der im letzten Kapitel entwickelte Anwendung führen Arbeiter einfache Berechnungen durch und aktualisieren nach jeder Iteration die Variablen auf dem Parameterserver. Bei dieser Anwendung verkürzt sich jedoch die Ausführungszeit nicht, da die Berechnung zu wenig Zeit in Anspruch nimmt und die Anzahl an Iterationen vorgegeben ist. Es ist aber zu sehen, dass wenn mehr als ein Arbeiter eingesetzt wird, sich mit jedem zusätzlichen Arbeiter sich das Ergebnis der Berechnung verbessert. D.h. wenn anstatt einem Arbeiter zwei Arbeiter verwendet werden würden, brächten wir pro Arbeiter nur die Hälfte an Iteration und somit halbe Zeit um das selbe Ergebnis zu erhalten.  