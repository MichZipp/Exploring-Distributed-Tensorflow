\section{TensorFlow}\label{sec:tensorflow}
TensorFlow ist eine Open-Source-Softwarebibliothek für die numerische Hochleistungsberechnung. Die flexible Architektur ermöglicht die einfache Implementierung von Berechnungen auf einer Vielzahl von Plattformen (\ac{cpu}s, \ac{gpu}s, TPUs) und von Desktops über Cluster von Servern bis hin zu mobilen und Edge-Geräten. Ursprünglich von Forschern und Ingenieuren des Google Brain-Teams in der KI-Organisation von Google entwickelt, bietet es eine starke Unterstützung für maschinelles Lernen und Deep Learning. Der Kern der flexiblen numerischen Berechnung wird in vielen anderen wissenschaftlichen Bereichen eingesetzt \cite{tensorflow}.

\subsection{Installation}
TensorFlow kann auf den folgenden Plattformversionen und höher installiert werden: 
\begin{itemize}
	\item macOS 10.12.6 
	\item Ubuntu 16.04 
	\item Windows 7 
\end{itemize}

Standardmäßig werden Anwendungen für TensorFlow in der Programmiersprache Python geschrieben, es lassen sich jedoch Entwicklungsumgebungen für die Programmiersprachen C++, Go, Swift und Java einrichten.

\subsection{Unterstützung von mobilen Plattformen}
Google stellt verschiedene Versionen ihres Frameworks bereit. Unter anderem auch Versionen für mobile Endgeräte, die für diese optimiert sind und mit weniger zur Verfügung stehenden Ressourcen auskommen. Im folgenden werden die verschiedenen Versionen kurz vorgestellt.

\subsubsection{TensorFlow Mobil}
TensorFlow Mobil wurde für den Einsatz auf einem mobilen Gerät optimiert und enthält alle Funktionen die das Framework zu bieten hat. Jedoch bietet TensorFlow Mobil nicht die optimale Performance für mobile Geräte. Deshalb wurde TensorFlow Light entwickelt, welches sich aktuell noch im Entwicklungsstatus befindet. Dabei werden von TensorFlow Light nur eine begrenzte Anzahl an Funktionen unterstützt, sodass die binäre Größe kleiner ist und damit ein Performance-Steigerung erzielt werden kann. Ob TensorFlow Mobil oder Light für eine Anwendung eingesetzt wird, hängt von deren Anforderungen ab \cite{tensorflow}. 

\subsubsection{TensorFlow Light}
TensorFlow Lite ist eine Lösung für mobile und eingebettete Geräte wie beispielsweise Smartphones oder Smartwatches. Es ermöglicht maschinelles Lernen auf dem Gerät mit geringer Wartezeit und einer kleinen binären Größe. TensorFlow Lite unterstützt auch Hardware-Beschleunigung mit der Android Neural Networks \ac{api}. TensorFlow Lite verwendet viele Techniken, um niedrige Latenzzeiten zu erreichen, wie z. B. die Optimierung der Kernel für mobile Apps, vorfusionierte Aktivierungen und quantisierte Kernel, die kleinere und schnellere Modelle ermöglichen \cite{tensorflow}. Gerade der Einsatz von einem Deep Learning Framework direkt auf einem mobilen Gerät ermöglicht viele neue Anwendungsfälle und erlaubt auch den Einsatz des Frameworks, wenn das mobile Gerät keine Internetverbindung hat. Meist reicht es auch aus, wenn auf dem mobilen Gerät ein bereits trainiertes Modell genutzt wird, sofern kein weiteres Training des Modells nötig ist. 

\subsubsection{TensorFlow.js}
TensorFlow.js ist eine JavaScript-Bibliothek zum Trainieren und Bereitstellen von Machine Learning Modellen im Browser und auf Node.js. Es können entweder vorhandene Modelle genutzt und weiter trainiert oder neue Modelle mit einer innovativen \ac{api} gebaut und bereitgestellt werden \cite{tensorflowjs}. Dies ermöglicht die Integration von maschinellem Lernen in Webseite, die auf jeder Plattform mit einem Browser genutzt werden kann.

