\section{Zusammenfassung}
 Eine \ac{ki} benötigt viele Ressourcen, um eine gute Performance zu gewährleisten. Es gibt heutzutage einige Deep Learning Frameworks, mit der eine \ac{ki} implementiert werden kann. Am Anfang des Artikels wurden einige Frameworks vorgestellt. Im Detail wurde auf das Framework TensorFlow von
 Google eingegangen, dabei wurden die unterstützen Plattformversionen und Möglichkeiten, wie TensorFlow auf einem mobilen Endgerät eingesetzt werden kann, beschrieben. Meist reicht eine Instanz nicht aus, um die Anforderungen einer Anwendungen zu erfüllen. Deshalb gibt es die Möglichkeit, mehrere Instanzen zu nutzen, um die Performance des Frameworks zu verbessern. Dafür wurde zuerst eine Wissensgrundlage geschaffen, indem die Architektur einer verteilten TensorFlow Umgebung aufgezeigt wurde. Anschließend wurde erläutert, welche Rolle dabei ein Arbeiter und Parameterserver spielt. Anhand eines Tutorials wurden die wichtigsten Schritte für das Aufsetzten einer solchen Umgebung beschrieben, wobei eine kleine Beispielanwendung entwickelt wurde. Diese wurde genutzt, um eine Evaluierung durchzuführen, welche bestätigt, dass der Einsatz von mehreren
 Arbeitern die Performance des Frameworks deutlich verbessert. Dabei wurden verschiedene Konfiguration, unterschiedliche Anzahl an Arbeiter und Parameterserver, für die TensorFlow Umgebung genutzt. Die Performance einer \ac{ki} wird auch in Zukunft eine Herausforderung darstellen, denn die Einsatzgebiete werden immer komplexer und breitgefächerter.  