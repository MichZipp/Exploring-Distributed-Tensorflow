\section{Einführung}
Eine \ac{ki} wird heutzutage immer mehr eingesetzt, um die Interaktion zwischen Mensch und Maschine zu bessern. Zusätzliche ermöglicht eine \ac{ki} vollkommen neue Möglichkeiten in Bereichen wie Robotik, Daten Analyse, Gesundheitswesen, Autonomes Fahren und vielen mehr. Doch was steckt hinter einer künstlichen Intelligenz? Eine künstliche Intelligenz versucht durch maschinelles Lernen die menschliche Wahrnehmung und das menschliche Handel durch Maschinen nachzubilden. Bei der \ac{ki}-Forschung gibt es viele Verbindung zur Neurologie und Psychologie, den das menschliche Denken muss erforscht und verstanden werden, bevor Maschinen dies nachahmen können. Bis heute ist dies noch nicht annähernd gelungen, die meisten \ac{ki} beschränken sich auf einen bestimmten Teilbereich und sind somit optimiert für einen bestimmten Anwendungsfall \cite{PlanetWissenKI}. \newline

Bei einer \ac{ki} kommt meist maschinelles Lernen zum Einsatz, wobei von erhobenen Daten gelernt wird, um dann Entscheidung treffen zu können. Es gibt eine Reihe an vordefinierten Algorithmen die an sogenannte Entscheidungsbäume erinnern. Diese Algorithmen sind meist nicht dynamisch genug, um mehrere Variablen verarbeiten zu können. Mit der Einführung von Deep Learning als Zweig des maschinellen Lernens ist dieses Problem größtenteils behoben. Durch Deep Learning kann ein kontinuierlicher Lernprozess umgesetzt werden, der sich stetig an neue Situation anpasst. Dabei basiert Deep Learning auf der Analyse von Big Data und gräbt sich durch riesige Mengen an Daten aus Datenbank, Internetquellen und mehr. Es nutzt neuronale Netze, um die Denkweise eines Menschen möglichst gut nachzuahmen. Deep Learning war bis vor wenigen Jahren nur von wenig Unternehmen genutzt, da der Einsatz viel Ressourcen benötigt und somit relativ teuer war. Durch den Fortschritt der Technik und der besseren Unterstützung der \ac{gpu} hat sich Deep Learning durchgesetzt und kommt in heutigen Durchbrüchen wie autonomes Fahren, maschinelle Übersetzung, Spracherkennung und vieles mehr zum Einsatz \cite{BigDataInsiderDeepLearning}. \newline

Aktuell stellt der Lernprozess in Echtzeit eine große Herausforderung dar, oftmals benötigt der Lernprozess mehrere Stunden bzw. Tage. Für einige Anwendungsfälle reicht dies nicht aus und es wird ein Lernprozess nahe zu in Echtzeit benötigt. Es gibt bereits zahlreiche Deep Learning Frameworks, welche in diesem Arikel aufgezeigt werden. Sehr beliebt ist aktuell das Framework TensorFlow von Google. Dieses Framework wird im Laufe dieses Artikel genauer untersucht und es wird anhand eines Tutorials die Einrichtung einer verteilten TensorFlow Umgebung aufgezeigt. Durch diese verteilte Umgebung kann die Performance des Lernprozesses verbessert werden.  
