\section{Einführung}
Eine \ac{ki} wird heutzutage immer mehr eingesetzt, um die Interaktion zwischen Mensch und Maschine zu verbessern. Zusätzliche ermöglicht eine \ac{ki} vollkommen neue Möglichkeiten in Bereichen wie Robotik, Daten-Analyse, Gesundheitswesen, Autonomes Fahren und vieles mehr. Doch was steckt hinter einer \ac{ki}? Eine \ac{ki} versucht durch maschinelles Lernen die menschliche Wahrnehmung und das menschliche Handeln durch Maschinen nachzubilden. Bei der \ac{ki}-Forschung gibt es viele Verbindungen zur Neurologie und Psychologie, da menschliches Denken erforscht und verstanden werden muss, bevor Maschinen dieses nachahmen können. Bis heute ist es noch nicht annähernd gelungen. Die meisten \ac{ki}s beschränken sich auf einen bestimmten Teilbereich und optimiert für einen bestimmten Anwendungsfall \cite{PlanetWissenKI}. \newline

Bei einer \ac{ki} kommt i.d.R. maschinelles Lernen zum Einsatz, wobei von erhobenen Daten gelernt wird, um daraus Entscheidungen zu treffen. Es gibt eine Reihe vordefinierter Algorithmen die an sogenannte Entscheidungsbäume erinnern. Diese Algorithmen sind jedoch nicht dynamisch genug, um mehrere Variablen verarbeiten zu können. Mit der Einführung von Deep Learning als Zweig des maschinellen Lernens ist dieses Problem größtenteils behoben. Durch Deep Learning kann ein kontinuierlicher Lernprozess umgesetzt werden, der sich stetig an neue Situationen anpasst. Dabei basiert Deep Learning auf der Analyse von Big Data und verarbeitet riesige Mengen an Daten aus Datenbank, Internetquellen und mehr. Es nutzt neuronale Netze, um die Denkweise eines Menschen möglichst gut nachzuahmen. Deep Learning wurde bis vor wenigen Jahren nur von wenigen Unternehmen genutzt, da der Einsatz viele Ressourcen benötigte und somit relativ teuer war. Durch den Fortschritt der Technik und der besseren Unterstützung der \ac{gpu} hat sich Deep Learning durchgesetzt und kommt in heutigen Anwendungen wie autonomes Fahren, maschinelle Übersetzung, Spracherkennung zum Einsatz \cite{BigDataInsiderDeepLearning}. \newline

Aktuell stellt der Lernprozess in Echtzeit eine große Herausforderung dar, oftmals benötigt dieser mehrere Stunden bzw. Tage. Für einige Anwendungsfälle dauert dies zu lange, da ein Lernprozess nahezu in Echtzeit benötigt wird. Es gibt bereits zahlreiche Deep Learning Frameworks, welche in diesem Arikel aufgezeigt werden. Sehr aktuell ist das Framework TensorFlow von Google. Dies wird im Laufe dieses Artikels genauer untersucht und es wird anhand eines Tutorials die Einrichtung einer verteilten TensorFlow Umgebung aufgezeigt. Durch diese verteilte Umgebung kann die Performance des Lernprozesses gesteigert werden.  
